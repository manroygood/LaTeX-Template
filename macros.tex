% These three definitions use \DeclarePairedDelimiter from mathtools
% Use with a * to get \left and \right
%\DeclarePairedDelimiter{\set}{\{}{\}}
\DeclarePairedDelimiterX{\innerp}[2]{\langle}{\rangle}{#1,#2}
\renewcommand{\ip}[2]{\innerp*{#1}{#2}}
%\DeclarePairedDelimiterX{\PB}[2]{\{}{\}}{#1,#2} % Poisson brackets

\newcommand{\normp}[2]{{\norm{#1}}_{#2}}

% We define a transpose command so we can code meaning not typesetting
\newcommand{\transpose[1]{ {#1}^{\mathsf{T}} }



% Shortcuts for some matrices, which are bold uppercase latin letters
\newcommand{\J}{\vb{J}}
\newcommand{\I}{\vb{I}}
\newcommand{\A}{\vb{A}}

% Shortcuts for some vectors, which are bold lowercase latin letters
\newcommand{\x}{\vb{x}}
\newcommand{\y}{\vb{y}}

% There is already a command \r, so we have to use \renewcommand instead of \newcommand
\renewcommand{\r}{\vb{r}}

% For use with the todonotes package. I create one of these for each author
\newcommand{\RG}[1]{\todo[inline,color=yellow]{RG: #1}}
\newcommand{\BMB}[1]{\todo[inline,color=green]{BMB: #1}}

% If you have words in a subscript, they should be typeset in Roman
\newcommand{\cLshort}{{\mathcal L}_{\textnormal{short}}}
\newcommand{\cLlong}{{\mathcal L}_{\textnormal{long}}}

% If you're using these often, you should make macros!
\newcommand{\CC}{{\mathbb C}}
\newcommand{\RR}{{\mathbb R}}
\newcommand{\ZZ}{{\mathbb Z}}

