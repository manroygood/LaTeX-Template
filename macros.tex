% These three definitions use \DeclarePairedDelimiter from mathtools
% Use with a * to get \left and \right
%\DeclarePairedDelimiter{\abs}{\lvert}{\rvert}
%\DeclarePairedDelimiter{\norm}{\lVert}{\rVert}
%\DeclarePairedDelimiter{\set}{\{}{\}}
\DeclarePairedDelimiterX{\innerp}[2]{\langle}{\rangle}{#1,#2}
\renewcommand{\ip}[2]{\innerp*{#1}{#2}}
%\DeclarePairedDelimiterX{\PB}[2]{\{}{\}}{#1,#2} % Poisson brackets

\newcommand{\normp}[2]{{\norm{#1}}_{#2}}

\newcommand{\Tee}{\mathsf{T}} % I use this for the transpose of a matrix

\newcommand{\Mf}{M^{\rm f}}
\newcommand{\Mb}{M^{\rm b}}

% Shortcuts for some matrices, which are bold uppercase latin letters
\newcommand{\J}{\mathbf{J}}
\newcommand{\I}{\mathbf{I}}
\newcommand{\A}{\mathbf{A}}

% Shortcuts for some vectors, which are bold lowercase latin letters
\newcommand{\x}{\mathbf{x}}
\newcommand{\y}{\mathbf{y}}
\newcommand{\q}{\mathbf{q}}
\newcommand{\p}{\mathbf{p}}
\newcommand{\z}{\mathbf{z}}

% There is already a command \r, so we have to use 
\renewcommand{\r}{\mathbf{r}}
\newcommand{\R}{\mathbf{R}}


% For use with the todonotes package. I create one of these for each author
\newcommand{\RG}[1]{\todo[inline,color=yellow]{RG: #1}}
\newcommand{\BMB}[1]{\todo[inline,color=green]{BMB: #1}}

% If you have words in a subscript, they should be typeset in Roman
\newcommand{\cLshort}{{\mathcal L}_{\textnormal{short}}}
\newcommand{\cLlong}{{\mathcal L}_{\textnormal{long}}}

% If you're using these often, you should make macros!
\newcommand{\CC}{{\mathbb C}}
\newcommand{\RR}{{\mathbb R}}
\newcommand{\ZZ}{{\mathbb Z}}

